\chapter{Introducci\'on}
\label{chap:intro}
Desde los primeros a\~{n}os en que el humano fabric\'o y utiliz\'o herramientas, la m\'usica ha formado parte de nuestras vidas, ya sea de forma directa o indirecta siempre habr\'a m\'usica. Inclusive cuando hablamos, a veces sentimos la necesidad de decir frases con cierta entonaci\'on que sigue un ritmo musical. No cabe duda que la m\'usica es importante para nosotros.

\noindent Muchos lo ven como una forma de distraerse del mundo, o como una forma de relajarse. Dependiendo de nuestro estado de \'animo buscaremos ciertos ritmos r\'apidos o lentos, fuertes o suaves, a veces buscando inspiraci\'on o concentraci\'on o simplemente escuchar m\'usica de pasatiempo.
\section{Motivaci\'on}
Desde un punto de vista m\'as t\'ecnico, la m\'usica es una composici\'on de sonidos que en conjunto siguen un ritmo espec\'ifico. Por alguna raz\'on buscamos escuchar sonidos que sigan un ritmo, pero preferimos evitar aquellos sonidos que carecen de el, algo que solemos llamar ruido.

\noindent Dependiendo del gusto de las personas, muchas consideran que ciertos g\'eneros de m\'usica solamente son ruido, mientras que otras los defienden. Los gustos var\'ian de persona a persona, toman influencia en aspectos como la forma en que nos educan, las personas con quien convivimos, las costumbres de la regi\'on, etc\'etera. Estos gustos a veces se ven alterados con el tiempo, factores como cambiar de regi\'on o las modas tambi\'en afectan los gustos.

\noindent Sin importar los g\'eneros musicales, el ruido es algo que perturba lo que escuchamos ya sea porque la calidad de audio es mala, o por los anuncios que alg\'un servicio incluya en su reproductor o simplemente las pausas entre pistas de audio.

\noindent Sabiendo que hay muchas variables que afectan lo que escuchamos, siempre se trata de llegar a un estado de armon�a a\'un sabiendo que cualquier cosa puede interrumpir este estado. ?`C\'omo evitar la interrupci\'on de la sensaci\'on que produce escuchar una pista de sonido? La respuesta es mezclar las pistas de audio en una sola usando la reproducci\'on sin pausas.
\section{Hip\'otesis}
\label{sec:hipo}
Es posible lograr mezclar el inicio y final de dos pistas de audio con ritmo similar mediante el uso de algoritmos de reconocimiento de patrones de tal forma que no se perciba el cambio entre melod\'ias para lograr una reproducci\'on sin pausas.

%Hay que evitar la interrupci\'on de la pista para dar una sensaci\'on de continuidad, pues las personas prefieren escuchar una sola melod\'ia que siga un ritmo sin interrupci\'on a escuchar varias melod\'ias con variaciones en su ritmo.

%Hay que considerar que la reproducci\'on sin pausa no siempre consiste en continuar r\'apidamente una pista despu\'es de al otra, ya que entre pistas de audio hay variaciones de ritmo. Es necesario mezclar los sonidos, es decir, buscar en la parte final de la primera pista y la parte inicial de la segunda un fragmento que sea compatible en ritmo, mezclarlos e iniciar una transici\'on suave para lograr un efecto de pista continua.

\section{Objetivos}
\label{sec:objs}
%La idea de este tipo de mezclado es similar al efecto llamado crossfading, pero se diferencia en que el crossfading mezcla independientemente del ritmo, b\'asicamente empieza una transici\'on suave entre los \'ultimos segundos de la primera pista y los primeros segundos de la segunda.

El objetivo principal es crear una herramienta capaz de reproducir archivos de sonido de forma continua, y mediante un algoritmo mezclar el inicio y final de las pistas reproducidas de forma que conserven su ritmo y no se perciba el cambio de pista.%que sea capaz de analizar patrones de sonido, compararlos entre las pistas y crear una reproducci\'on continua al mezclar los fragmentos entre las pistas mediante una transici\'on. 

\noindent Los objetivos espec\'ificos son:
\begin{itemize}
\item Crear un reproductor de audio que haga uso de esta herramienta para crear una transici\'on casi imperceptible entre pistas.
\item Realizar pruebas con usuarios con la m\'usica de su gusto.
\item Tratar de implementar la herramienta a alguna otra plataforma ya sea en una aplicaci\'on m\'ovil o una aplicaci\'on web.
\end{itemize}

%Esto abre camino a una gran cantidad de aplicaciones que utilicen el mismo principio como:

%\begin{itemize}
%\item Software capaz de crear mezclas de m\'usica de forma inteligente.
%\item Sistemas de predicci\'on, capaces de crear listas de reproducci\'on basado en an\'alisis previos.
%\item Sistemas de b\'usqueda de m\'usica basada en el ritmo.
%\end{itemize}

\section{Estructura de la tesis}

El presente trabajo de tesis est\'a organizado de la siguiente manera:

\noindent En el cap\'itulo \ref{chap:intro} se habla en forma general sobre la intenci\'on de la tesis y se introducen algunos conceptos en la secci\'on de hip\'otesis y objetivos que son utilizados durante la implementaci\'on del proyecto de tesis.

\noindent En el cap\'itulo \ref{chap:antec} se habla de los antecedentes a los temas que se abordan. Se definen conceptos generales y espec\'ificos a los temas {\em procesamiento de se\~nales} y {\em reconocimiento de patrones}.

\noindent En el cap\'itulo \ref{chap:estart} se presentan trabajos que de alguna manera se relacionan con el tema y proyecto de tesis. Los trabajos presentados se dividen en aquellos que investigan sobre la detecci\'on r\'itmica en una pista de sonido y aquellos que implementan alguna herramienta o aplicaci\'on utilizando la misma.

\noindent En el cap\'itulo \ref{chap:sol} se define la metodolog\'ia; se mencionan las herramientas utilizadas y se muestra el desarrollo del trabajo realizado en cuanto a procesamiento de se\~nales y reconocimiento de patrones. Tambi\'en se muestra el dise\~no e implementaci\'on del software, los datos procesados y la forma en que se trabajan los datos.

\noindent En el cap\'itulo \ref{chap:eval} se eval\'ua el desempe\~no del software desarrollado bajo ciertos par\'ametros, se muestran resultados de pruebas con usuarios  y finalmente se tiene una discusi\'on acerca de lo bueno y lo malo de la implementaci\'on y el algoritmo as\'i como del grado de aceptaci\'on por parte de los usuarios hacia el proyecto.

\noindent En el cap\'itulo \ref{chap:conc} se resumen los resultados obtenidos con el proyecto. Tambi\'en se habla sobre las \'areas de oportunidad del mismo y el trabajo a futuro.