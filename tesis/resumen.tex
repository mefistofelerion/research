%Resumen

\chapter*{Resumen}
\markboth{Resumen}{}

\noindent\autor.

\noindent Candidato para el grado de Ingeniero en Tecnolog\'{\i}a de Software.
%\indent 

\noindent\uanl.\\
\noindent\fime.

\noindent T\'{\i}tulo del estudio:

\begin{center}
\begin{tabular}{p{11cm}}
	\centering
	\scshape{\large{\titulo}}
\end{tabular}
\end{center}\bigskip

\noindent N\'{u}mero de p\'{a}ginas: \pageref*{lastpage}.

\paragraph{Objetivos y m\'{e}todo de estudio:}

El objetivo de este trabajo es desarrollar un software que utilice t\'ecnicas de an\'alisis de se\~nales para reconocer ciertos patrones sonoros en la m\'usica, de esta manera generar una lista de reproducci\'on basada en el ritmo obtenido de los archivos reproducidos. Adicionalmente se desea experimentar con efectos de transici\'on entre pistas musicales para dar un efecto de reproducci\'on sin pausa entre pistas que aparentemente tienen un ritmo parecido, diferenciar y detectar el principio y final de una pista, as\'i como el momento \'optimo para realizar la transici\'on entre pistas.

\noindent El desarrollo de software consiste en su mayor parte de an\'alisis de informaci\'on obtenida de archivos de sonido, el sistema trabaja en tiempo real, por lo que todo ocurre mientras se reproduce el sonido. La intenci\'on es conocer la opini\'on de los usuarios en cuanto a eficiencia.

\paragraph{Contribuciones y conclusiones:}

La contribuci\'on principal de este trabajo es la aplicaci\'on de algoritmos de reconocimiento de patrones y t\'ecnicas de an\'alisis de se\~nales en un software para lograr una reproducci\'on musical sin pausas, con la intenci\'on de mejorar la experiencia musical del usuario.

\noindent El programa generado fue probado bajo ciertas circunstancias de software y hardware para obtener un an\'alisis de funcionamiento detallado y que servir\'a para mejorar a futuro.

\noindent Firma del asesor: \rule{78mm}{0.3pt}

\vspace*{-4mm}

\noindent \phantom{Firma del asesor: m} \asesor
